\definecolor{textcolor}{rgb}{0, 0, 0}

\newkeycommand{\reactor}[x = 0, y = 0, width = 2cm, height = 3cm, archeight = 1cm, name = reactor, level = -1cm, textpos = 0cm][1]{
    \draw
    (\commandkey{x} - 0.5*\commandkey{width}, \commandkey{y} - 0.5*\commandkey{height} + \commandkey{archeight}) to[out = -90, in = 180]
    (\commandkey{x}, \commandkey{y} - 0.5*\commandkey{height}) to[out = 0, in = -90]
    (\commandkey{x} + 0.5*\commandkey{width}, \commandkey{y} - 0.5*\commandkey{height} + \commandkey{archeight}) --
    (\commandkey{x} + 0.5*\commandkey{width}, \commandkey{y} + 0.5*\commandkey{height} - \commandkey{archeight}) to[out = 90, in = 0]
    (\commandkey{x}, \commandkey{y} + 0.5*\commandkey{height}) to[out = 180, in = 90]
    (\commandkey{x} - 0.5*\commandkey{width}, \commandkey{y} + 0.5*\commandkey{height} - \commandkey{archeight}) --
    cycle;
    \node[rectangle, minimum width = \commandkey{width}, minimum height = \commandkey{height}] (\commandkey{name}) at (\commandkey{x}, \commandkey{y}){};

    \node[align = center] at (\commandkey{x}, \commandkey{y} - 0.5*\commandkey{height} + 0.5*\commandkey{archeight} + \commandkey{textpos}) {#1};

    \begin{scope}
      \clip (\commandkey{x} - 0.5*\commandkey{width}, \commandkey{y} - 0.5*\commandkey{height} + \commandkey{archeight}) to[out = -90, in = 180]
        (\commandkey{x}, \commandkey{y} - 0.5*\commandkey{height}) to[out = 0, in = -90]
        (\commandkey{x} + 0.5*\commandkey{width}, \commandkey{y} - 0.5*\commandkey{height} + \commandkey{archeight}) --
        (\commandkey{x} + 0.5*\commandkey{width}, \commandkey{y} + 0.5*\commandkey{height} - \commandkey{archeight}) to[out = 90, in = 0]
        (\commandkey{x}, \commandkey{y} + 0.5*\commandkey{height}) to[out = 180, in = 90]
        (\commandkey{x} - 0.5*\commandkey{width}, \commandkey{y} + 0.5*\commandkey{height} - \commandkey{archeight}) --
        cycle;
      \draw[blue, decorate, decoration={snake}]
      (\commandkey{x} - \commandkey{width}, \commandkey{y} - 0.5*\commandkey{height} + \commandkey{level}) --
      (\commandkey{x} + \commandkey{width}, \commandkey{y} - 0.5*\commandkey{height} + \commandkey{level});
    \end{scope}
}

\newkeycommand{\bus}[x = 0, y = 0, width = 0.5cm, length = 5cm, inner sep = 2pt, name = bus][1]{
    \draw
    (\commandkey{x} - 0.5*\commandkey{length}, \commandkey{y} + 0.5*\commandkey{width}) --
    (\commandkey{x} + 0.5*\commandkey{length}, \commandkey{y} + 0.5*\commandkey{width})
    to [out = 0, in = 0]
    (\commandkey{x} + 0.5*\commandkey{length}, \commandkey{y} - 0.5*\commandkey{width}) --
    (\commandkey{x} - 0.5*\commandkey{length}, \commandkey{y} - 0.5*\commandkey{width})
    to [out = 180, in = 180] cycle;

    \draw
    (\commandkey{x} + 0.5*\commandkey{length}, \commandkey{y} + 0.5*\commandkey{width} - \commandkey{inner sep}) to [out = 0, in = 0]
    (\commandkey{x} + 0.5*\commandkey{length}, \commandkey{y} - 0.5*\commandkey{width} + \commandkey{inner sep}) to [out = 180, in = 180] cycle;

    \node[rectangle, minimum width = \commandkey{length}, minimum height = \commandkey{width}] (\commandkey{name}) at (\commandkey{x}, \commandkey{y}) {#1};
}

\newkeycommand{\computer}[x = 0cm, y = 0cm, width = 2cm, height = 2cm, name = computer][1]{
    \node at (\commandkey{x}, \commandkey{y}) {\includegraphics[width=\commandkey{width}]{Figures/Materials/Computer.pdf}};

    \node[rectangle, minimum width = 0.82*\commandkey{width}, minimum height = 0.82*\commandkey{height}] (\commandkey{name}) at (\commandkey{x}, \commandkey{y}){};

    \shadowtext[align=left, anchor = north west, inner sep = 0pt] at (\commandkey{x} - 0.5*\commandkey{width}, \commandkey{y} + 0.5*\commandkey{height}) {#1};
}

\newkeycommand{\server}[x = 0cm, y = 0cm, width = 1.35cm, height = 2cm, name = server][1]{
    \node at (\commandkey{x}, \commandkey{y}) {\includegraphics[width=\commandkey{width}]{Figures/Materials/Server.pdf}};

    \node[rectangle, minimum width = 1.16*\commandkey{width}, minimum height = 0.76*\commandkey{height}] (\commandkey{name}) at (\commandkey{x}, \commandkey{y}){};

    \shadowtext[align=left, anchor = north west, inner sep = 0pt] at (\commandkey{x} - 0.5*\commandkey{width}, \commandkey{y} + 0.5*\commandkey{height}) {#1};
}

\newkeycommand{\plc}[x = 0cm, y = 0cm, width = 1.8cm, height = 2cm, name = plc][1]{
    \node at (\commandkey{x}, \commandkey{y}) {\includegraphics[width=\commandkey{width}]{Figures/Materials/PLC.pdf}};

    \node[rectangle, minimum width = 0.76*\commandkey{width}, minimum height = 0.76*\commandkey{height}] (\commandkey{name}) at (\commandkey{x}, \commandkey{y}){};

    \shadowtext[align=center, inner sep = 0pt] at (\commandkey{x}, \commandkey{y}) {#1};
}

\newkeycommand{\gateway}[x = 0cm, y = 0cm, width = 1.8cm, height = 1.5cm, name = gateway][1]{
    \node at (\commandkey{x}, \commandkey{y}) {\includegraphics[width=\commandkey{width}]{Figures/Materials/Gateway.pdf}};

    \node[rectangle, minimum width = \commandkey{width}, minimum height = 0.74*\commandkey{height}] (\commandkey{name}) at (\commandkey{x}, \commandkey{y} - 0.08cm){};

    \shadowtext[align=center, inner sep = 0pt] at (\commandkey{x}, \commandkey{y}) {#1};
}

\newkeycommand{\valve}[x = 0cm, y = 0cm, width = 0.6cm, height = 0.4cm, name = valve][1]{
    \node[rectangle, minimum width = \commandkey{width}, minimum height = 2.3*\commandkey{height}, inner sep = 0pt] (\commandkey{name}) at (\commandkey{x}, \commandkey{y}){};
    \draw (\commandkey{x} + 0.5*\commandkey{width}, \commandkey{y} + 0.5*\commandkey{height}) --
          (\commandkey{x} + 0.5*\commandkey{width}, \commandkey{y} - 0.5*\commandkey{height}) --
          (\commandkey{x} - 0.5*\commandkey{width}, \commandkey{y} + 0.5*\commandkey{height}) --
          (\commandkey{x} - 0.5*\commandkey{width}, \commandkey{y} - 0.5*\commandkey{height}) -- cycle;
    \draw (\commandkey{x}, \commandkey{y}) --
          (\commandkey{x}, \commandkey{y} + 0.85*\commandkey{height}) --
          (\commandkey{x} - 0.5*\commandkey{width}, \commandkey{y} + 0.85*\commandkey{height})
          to [out = 45, in = 135]
          (\commandkey{x} + 0.5*\commandkey{width}, \commandkey{y} + 0.85*\commandkey{height}) --
          (\commandkey{x}, \commandkey{y} + 0.85*\commandkey{height});
    \node[below = 5pt] at (\commandkey{x}, \commandkey{y}) {#1};
}

\newkeycommand{\switch}[x = 0cm, y = 0cm, width = 0.5cm, name = switch][1]{
    \node[rectangle, minimum width = \commandkey{width}, minimum height = 0.5774*\commandkey{width}, inner sep = 0pt] (\commandkey{name}) at (\commandkey{x}, \commandkey{y}) {};
    \draw (\commandkey{x} - 0.5*\commandkey{width}, \commandkey{y}) -- ++(30:\commandkey{width});
    \node[draw, circle, minimum size = 1pt, fill = white, inner sep = 0pt] at (\commandkey{x} - 0.5*\commandkey{width}, \commandkey{y}) {};
    \node[below = 0pt] at (\commandkey{x}, \commandkey{y}) {#1};
}

\newkeycommand{\motor}[x = 0cm, y = 0cm, width = 0.7cm, name = motor][1]{
    \node[draw, circle, minimum size = \commandkey{width}, inner sep = 0pt] (\commandkey{name}) at (\commandkey{x}, \commandkey{y}) {#1};
}

\newkeycommand{\sensor}[x = 0cm, y = 0cm, width = 0.18cm, height = 0.5cm, name = sensor][1]{
    \node[rectangle, inner sep = 0pt, minimum width = \commandkey{width}, minimum height = \commandkey{height}, fill = black] (\commandkey{name}) at (\commandkey{x}, \commandkey{y}) {};

    \shadowtext[align=center, inner sep = 0pt, below = 0.7*\commandkey{height}] at (\commandkey{x}, \commandkey{y}) {#1};
}

\newkeycommand{\omission}[x = 0cm, y = 0cm, angle = 0, length = 0.4cm, name = omission]{
    \begin{scope}
    \clip[rotate around={\commandkey{angle}:(\commandkey{x}, \commandkey{y})}] (\commandkey{x} - 0.45*\commandkey{length} , \commandkey{y} - 0.45*\commandkey{length} ) rectangle (\commandkey{x} + 0.45*\commandkey{length}  , \commandkey{y} + 0.45*\commandkey{length} );
    \foreach \x/\c in {1pt/white,
                       2pt/black,
                       3pt/white,
                       4pt/black,
                       5pt/white}{
    \draw[line width = 1pt, draw = \c] (\commandkey{x}, \commandkey{y} - 3pt + \x) to[out = 120 + \commandkey{angle}, in = 60 + \commandkey{angle}] ++(180 + \commandkey{angle}:0.5*\commandkey{length})
        (\commandkey{x}, \commandkey{y} - 3pt + \x) to[out = -60 + \commandkey{angle}, in = -120 + \commandkey{angle}] ++(\commandkey{angle}:0.5*\commandkey{length});

    }
    \end{scope}
}

\newkeycommand{\coverpipe}[x = 0cm, y = 0cm, length = 1cm, name = coverpipe]{
    \draw[dashed, line width = 4pt, draw = white, dash pattern = on 2pt off 3pt] (\commandkey{x} - 0.5*\commandkey{length}, \commandkey{y}) -- ++ (0:\commandkey{length});
    \draw (\commandkey{x} - 0.5*\commandkey{length}, \commandkey{y} + 3pt) -- (\commandkey{x} - 0.5*\commandkey{length}, \commandkey{y} - 3pt);
    \draw (\commandkey{x} + 0.5*\commandkey{length}, \commandkey{y} + 3pt) -- (\commandkey{x} + 0.5*\commandkey{length}, \commandkey{y} - 3pt);

    \draw[white] (\commandkey{x} - 0.5*\commandkey{length} - 1pt, \commandkey{y} + 3pt) -- (\commandkey{x} - 0.5*\commandkey{length} - 1pt, \commandkey{y} - 3pt);
    \draw[white] (\commandkey{x} + 0.5*\commandkey{length} + 1pt, \commandkey{y} + 3pt) -- (\commandkey{x} + 0.5*\commandkey{length} + 1pt, \commandkey{y} - 3pt);
}

\newkeycommand{\blender}[x = 0cm, y = 0cm, length = 2cm, name = blender][1]{
    \coordinate (\commandkey{name}) at (\commandkey{x}, \commandkey{y});
%    \fill (\commandkey{x}, \commandkey{y})
%    to [out = 30, in = 150] ++ (0:0.5*\commandkey{length})
%    to [out = -150, in = -30] (\commandkey{x}, \commandkey{y})
%    to [out = 150, in = 30] ++ (180:0.5*\commandkey{length})
%    to [out = -30, in = -150] (\commandkey{x}, \commandkey{y});
    \node[fill = black, rectangle, inner sep = 0pt, minimum width = 0.5*\commandkey{length}, minimum height = 0.3cm] (\commandkey{name}L) at (\commandkey{x} - 0.25*\commandkey{length} - 0.2cm, \commandkey{y}) {};
    \node[fill = black, rectangle, inner sep = 0pt, minimum width = 0.5*\commandkey{length}, minimum height = 0.3cm] (\commandkey{name}R) at (\commandkey{x} + 0.25*\commandkey{length} + 0.2cm, \commandkey{y}) {};
    \node[below = 0.13cm] at (\commandkey{name}) {#1};
%    \draw[line width = 3pt, white] (\commandkey{name}) -- (\commandkey{name}L.center) -- ++(180:1pt);;
%    \draw[line width = 3pt, white] (\commandkey{name}) -- (\commandkey{name}R.center) -- ++(0:1pt);
    \draw (\commandkey{name}R.center) -- (\commandkey{name}L.center);
}

\newkeycommand{\internet}[x = 0cm, y = 0cm, width = 2cm, name = internet][1]{
    \node[draw, cloud, minimum width = \commandkey{width}, minimum height = 2cm, inner sep = 0pt, align = center] (\commandkey{name}) at (\commandkey{x}, \commandkey{y}) {};
    \node at (\commandkey{x}, \commandkey{y}) {\parbox{\commandkey{width}}{\centering #1}};
}

\newkeycommand{\heater}[x = 0cm, y = 0cm, width = 0.5cm, coverpos = 0.5cm, length = 2cm, name = heater][1]{
    \node[rectangle, minimum width = \commandkey{width}, minimum height = 0.5774*\commandkey{width}, inner sep = 0pt] (\commandkey{name}) at (\commandkey{x}, \commandkey{y}) {};
    \draw (\commandkey{x} - 0.5*\commandkey{width}, \commandkey{y}) -- ++(30:\commandkey{width});
    \node[draw, circle, minimum size = 1pt, fill = white, inner sep = 0pt] at (\commandkey{x} - 0.5*\commandkey{width}, \commandkey{y}) {};
    \node[below = 0pt] at (\commandkey{x}, \commandkey{y}) {#1};

    \draw[line width = 1pt, draw = white, double=black, double distance = 1pt]
    (\commandkey{name}) --
    ++(180:\commandkey{length}) to[resistor ={minimum width = 0.8cm, minimum height = 0.1cm}]
    ++(-90:1cm) --
    ++(0:2.5cm + \commandkey{length}) to[ac source = {rotate = 90, minimum size = 0.7cm}]
    ++(90:1cm) -- (\commandkey{name});
}
\begin{tikzpicture}[circuit ee IEC,
                    set resistor graphic=var resistor IEC graphic,
                    line width = 1pt,
                    % x = 1cm,
                    % y = 1cm,
                    pipe/.style = {line width = 1pt, draw = white, double=black, double distance = 3pt},
                    wire/.style = {line width = 1pt, draw = blue}]

\linespread{0.9}
\small
% Draw help lines.
% \draw (0, 0) to[grid with coordinates]  (12, 17);

% Draw three networks.
\bus     [x = 6.00cm, y = 14.5cm, name = Ethernet, length = 11.6cm]{Ethernet};
\bus     [x = 3.00cm, y = 11.5cm, name = CANBUS1, length = 5.6cm]{CANBUS};
\bus     [x = 9.00cm, y = 11.5cm, name = CANBUS2, length = 5.6cm]{CANBUS};

% Draw devices in the Ethernet.
\computer[x = 1.00cm, y = 16.0cm, name = ES]{Engineer\\Station};
\server  [x = 4.33cm, y = 16.0cm, name = HDS]{Historical\\Data Server};
\gateway [x = 7.66cm, y = 16.0cm, name = GW]{Gateway of\\Ethernet};
\internet[x = 11.0cm, y = 16.0cm, name = EN]{Enterprise\\Network};

% Draw gateways of two CANBUS networks.
\gateway [x = 3.00cm, y = 13.0cm, name = GW1]{Gateway of\\CANBUS};
\gateway [x = 9.00cm, y = 13.0cm, name = GW2]{Gateway of\\CANBUS};

% Draw PLCs in the CANBUS networks.
\foreach \i in {1,2,...,6}{
    \plc [x = 2*\i cm - 1cm, y = 10cm, name = PLC\i]{PLC\i};
}

% Draw wires.
\draw (Ethernet.north -| ES) -- (ES);
\draw (Ethernet.north -| HDS) -- (HDS);
\draw (Ethernet.north -| GW) -- (GW);
\draw (GW) -- (EN);

\draw (Ethernet.south -| GW1) -- (GW1);
\draw (Ethernet.south -| GW2) -- (GW2);
\draw (CANBUS1.north -| GW1) -- (GW1);
\draw (CANBUS2.north -| GW2) -- (GW2);
\foreach \i in {1, 2, ..., 6}{
    \ifthenelse{\i < 4}
    {
        \draw (CANBUS1.south -| PLC\i) -- (PLC\i);
    }
    % else
    {
        \draw (CANBUS2.south -| PLC\i) -- (PLC\i);
    }
}

% Draw chemical reactor.
\reactor [x =  6.0cm, y = 3cm, height = 8.0cm, width =  4.0cm, level = 5.0cm, name = Reactor, textpos = 1.75cm]{Continuous\\Stirred-Tank Reactor};

% Draw four valves.
\valve   [x =  2.0cm, y =  5.0cm, name = V1 ]{V1};
\valve   [x =  2.0cm, y =  3.0cm, name = V2 ]{V2};
\valve   [x = 10.0cm, y =  3.0cm, name = V3 ]{V3};
\valve   [x = 10.0cm, y =  5.0cm, name = V4 ]{V4};

% Draw blender and its motor.
\motor   [x =  6.0cm, y =  8.0cm, name = M]{M};
\blender [x =  6.0cm, y =  2.5cm, name = B]{Impeller};

% Draw three sensors.
\sensor  [x =  4.5cm, y =  4.5cm, name = PS]{PS};
\sensor  [x =  5.0cm, y =  3.5cm, name = LS]{LS};
\sensor  [x =  5.5cm, y =  3.5cm, name = TS]{TS};

% Draw heater.
\heater  [x =  9cm, y =  0.5cm, name = SW, length = 3cm]{SW};

% Draw white covers.
\foreach \x/\y in {4.5/6.7,
                   5.0/6.9,
                   5.5/7.0,
                   5.0/4.0,
                   5.5/4.0,
                   6.0/7.0,
                   6.0/4.0}{
    \draw[line width = 3pt, white] (\x, \y - 0.2) -- (\x, \y + 0.2);
}

% Draw signal wires.
\draw[wire]  (PLC1) -- ++(-90:4cm) -| (V1)
             (PLC1) -- ++(-90:6cm) -| (V2);

\draw[wire]  (PLC2) -- ++(-90:2.5cm) -| (PS)
             (PLC2) -- ++(-90:2.5cm) -| (LS)
             (PLC2) -- ++(-90:2.5cm) -| (TS);

\draw[wire]  (PLC3) |- (M);
\draw        (M) -- (B);

\draw[wire]  (PLC4) -- ++(-90:2.5cm) -| (SW);
\draw[wire]  (PLC5) -- ++(-90:2cm) -| (V4);
\draw[wire]  (PLC6) -- ++(-90:6cm) -| (V3);

% Draw pipes.
\draw[pipe]  (Reactor.west |- V1) -- (V1) -- ++(180:1.9cm);
\draw[pipe]  (Reactor.west |- V2) -- (V2) -- ++(180:1.9cm);
\draw[white] (V1) -- ++(0:2.1cm)
             (V1) -- ++(180:2cm)
             (V2) -- ++(0:2.1cm)
             (V2) -- ++(180:2cm);

\draw[pipe]  (Reactor.east |- V3) -- (V3) -- ++(0:1.9cm);
\draw[pipe]  (Reactor.east |- V4) -- (V4) -- ++(0:1.9cm);
\draw[white] (V3) -- ++(180:2.1cm)
             (V3) -- ++(0:2cm)
             (V4) -- ++(180:2.1cm)
             (V4) -- ++(0:2cm);

% Draw legend.
\fill[black] (0.2, 1.9) rectangle (3.8, -1);
\draw[draw = black, fill = white] (0.1, 2) rectangle (3.7, -0.9);

\tabulinesep = 2.3pt
\node[font = \footnotesize] at (2.1, 0.5) {
    \begin{tabu}to 3.7cm{@{}X[-1]@{\hskip 0.2cm}X}
        PS & Pressure Sensor\\
        LS & Liquid Level Sensor\\
        TS & Temperature Sensor\\
        SW & Switch of Heater\\
        M  & Motor of Impeller\\
        V  & Valve\\
    \end{tabu}
};

\node[inner sep = 1pt, anchor = west, fill = white, text = blue] at (0.2, 2) {Legend};
\end{tikzpicture} 