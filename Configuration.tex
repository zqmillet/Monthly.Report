%!TEX root = ./Main.tex

\usepackage{Theme/BeamerTheme}
\usefonttheme[onlymath]{serif}
\usepackage{hyperref}
\hypersetup{pdfpagemode=FullScreen}
% \usepackage{silence}
% \WarningFilter{latex}{You have requested package}

\usepackage{qrcode}
\usepackage{amsmath}
\usepackage{mathrsfs}
\usepackage{mathtools}
\usepackage{commath}
\usepackage{amssymb}
\usepackage{bm}
\usepackage{listings}
\usepackage{mathtools}
\usepackage{booktabs}
\usepackage{multirow}
\usepackage{ragged2e}
\justifying
\let\raggedright\justifying
\usepackage{tikz}
\usepackage{tkz-euclide}
\usetikzlibrary{shadows}
\usetikzlibrary{positioning}
\usetikzlibrary{circuits.ee.IEC}
\usetikzlibrary{decorations.pathmorphing}
\usetikzlibrary{shapes}
\usetkzobj{all}
\usepackage[framed,numbered]{matlab-prettifier}
\usepackage{pgfplots}
\pgfplotsset{compat=1.3}
\usepgfplotslibrary{groupplots}
\usepgfplotslibrary{fillbetween}

\usepackage{tabu}
\usepackage{multirow}
\usepackage{graphicx}
\usepackage{colortbl}
\usepackage{ifthen}
\usepackage{siunitx}
\ExplSyntaxOn
\cs_new_eq:NN \siunitx_table_collect_begin:Nn \__siunitx_table_collect_begin:Nn
\ExplSyntaxOff

\usepackage{soul}
\usepackage{keycommand}

\DeclareMathSizes{5}{5}{3}{3}

\newcommand{\pgfdefaultlinewidth}{0.75pt}

\makeatletter
\lstnewenvironment{bash}[1][]{%
  \lstset{%
    #1,
    style = Matlab-editor,
    numbers=left,
    numberstyle = \scriptsize,
    basicstyle = \ttfamily\footnotesize,
    firstnumber=auto,
  }%
  \csname\@lst @SetFirstNumber\endcsname
}{%
  \csname \@lst @SaveFirstNumber\endcsname
}
\makeatother

% The following codes are used to draw help lines for tikzpicture.
\makeatletter
\def\grd@save@target#1{%
  \def\grd@target{#1}}
\def\grd@save@start#1{%
  \def\grd@start{#1}}
\tikzset{
  grid with coordinates/.style={
    to path={%
      \pgfextra{%
        \edef\grd@@target{(\tikztotarget)}%
        \tikz@scan@one@point\grd@save@target\grd@@target\relax
        \edef\grd@@start{(\tikztostart)}%
        \tikz@scan@one@point\grd@save@start\grd@@start\relax
        \draw[minor help lines] (\tikztostart) grid (\tikztotarget);
        \draw[major help lines] (\tikztostart) grid (\tikztotarget);
        \grd@start
        \pgfmathsetmacro{\grd@xa}{\the\pgf@x/1cm}
        \pgfmathsetmacro{\grd@ya}{\the\pgf@y/1cm}
        \grd@target
        \pgfmathsetmacro{\grd@xb}{\the\pgf@x/1cm}
        \pgfmathsetmacro{\grd@yb}{\the\pgf@y/1cm}
        \pgfmathsetmacro{\grd@xc}{\grd@xa + \pgfkeysvalueof{/tikz/grid with coordinates/major step}}
        \pgfmathsetmacro{\grd@yc}{\grd@ya + \pgfkeysvalueof{/tikz/grid with coordinates/major step}}
        \foreach \x in {\grd@xa,\grd@xc,...,\grd@xb}
        \node[anchor=north] at (\x,\grd@ya) {\pgfmathprintnumber{\x}};
        \foreach \y in {\grd@ya,\grd@yc,...,\grd@yb}
        \node[anchor=east] at (\grd@xa,\y) {\pgfmathprintnumber{\y}};
      }
    }
  },
  minor help lines/.style={
    help lines,
    step=\pgfkeysvalueof{/tikz/grid with coordinates/minor step}
  },
  major help lines/.style={
    help lines,
    line width=\pgfkeysvalueof{/tikz/grid with coordinates/major line width},
    step=\pgfkeysvalueof{/tikz/grid with coordinates/major step}
  },
  grid with coordinates/.cd,
  minor step/.initial=.2,
  major step/.initial=1,
  major line width/.initial=2pt,
}
\makeatother


\tikzset{circuit declare symbol = ac source}
% \tikzset{font = \fontsize{10pt}{12pt}\selectfont}
\tikzset{set ac source graphic = ac source IEC graphic}
\tikzset
{
    ac source IEC graphic/.style=
    {
        transform shape,
        circuit symbol lines,
        circuit symbol size = width 3 height 3,
        shape=generic circle IEC,
        /pgf/generic circle IEC/before background=
        {
            \pgfpathmoveto{\pgfpoint{-0.8pt}{0pt}}
            \pgfpathsine{\pgfpoint{0.4pt}{0.4pt}}
            \pgfpathcosine{\pgfpoint{0.4pt}{-0.4pt}}
            \pgfpathsine{\pgfpoint{0.4pt}{-0.4pt}}
            \pgfpathcosine{\pgfpoint{0.4pt}{0.4pt}}
            \pgfusepathqstroke
        }
    }
}

\def\shadowtext[#1]#2at#3(#4)#5{
    \foreach \angle in {0,5,...,359}{
    	\node[#1, text=white] at ([shift={(\angle:.8pt)}] #4){#5};
    }
    \node[#1] at (#4){#5};
}


\newtheorem{prop}{Proposition}

% The follow codes are macros for frequent terms.

\newif\ifCompleteCompile
\CompleteCompiletrue