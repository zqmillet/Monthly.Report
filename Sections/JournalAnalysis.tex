%!TEX root = ../Main.tex

\newcommand{\distributionofRT}[6]{
\begin{frame}{Distribution of Review Time (#6 Papers)}
  Journal: #2.
  \begin{flushleft}
    \begin{tikzpicture}
      \begin{axis}[bar width   = 0.3cm,
                   width       = \textwidth,
                   height      = 6cm,
                   ylabel      = Paper Number,
                   xlabel      = Review Time (Day),
                   label style = {font = \footnotesize}]
        \addplot[ybar, fill = black, draw = none] file {./JournalAnalysis/OutputData/#1ReviewTimeDistribution.dat};
      \end{axis}
    \end{tikzpicture}
  \end{flushleft}\vspace{-10pt}
  Minimum: #3,\hspace{\fill} Average: #4,\hspace{\fill} Maximum: #5.
\end{frame}
}

\newcommand{\distributionofPN}[6]{
\begin{frame}{Distribution of Page Number (#6 Papers)}
  Journal: #2.
  \begin{flushleft}
    \begin{tikzpicture}
      \begin{axis}[bar width   = 0.83cm,
                   width       = \textwidth,
                   height      = 6cm,
                   ylabel      = Paper Number,
                   xlabel      = Page Number,
                   label style = {font = \footnotesize}]
        \addplot[ybar, fill = black, draw = none] file {./JournalAnalysis/OutputData/#1PageNumberDistribution.dat};
      \end{axis}
    \end{tikzpicture} 
    \end{flushleft}\vspace{-10pt}
    Minimum: #3,\hspace{\fill} Average: #4,\hspace{\fill} Maximum: #5.
\end{frame}
}

\newcommand{\relationshipbetweenRTPN}[3]{
\begin{frame}{Scatter of Review Time and Page Number (#3 Papers)}
  Journal: #2.
  \begin{flushleft}
    \begin{tikzpicture}
      \begin{axis}[width       = \textwidth,
                   height      = 6.4cm,
                   ylabel      = Page Number,
                   xlabel      = Review Time (Day),
                   label style = {font = \footnotesize}]
        \addplot[scatter, only marks, scatter/use mapped color = 
        {draw opacity = 0, fill = mapped color} ] table[x index = 1, y index = 2, col sep=space] {./JournalAnalysis/Journals/#1/References.dat};        
      \end{axis}
    \end{tikzpicture}
  \end{flushleft}
\end{frame}
}

\section{Analysis of Journals}
\subsection{Proceedings of the IEEE}
\begin{frame}{About Journal}
  \textbf{Proceedings of the IEEE} (ISSN: 0018-9219), established in 1909, is one the oldest journals of IEEE. This journal focuses on electrical engineering and computer science. The other information is shown in the following table.\vspace{-10pt}
  \begin{center}
    \begin{tabu}to \textwidth{*4{X[-1, c]}}
    \tabucline[1pt]{-}
      Year      & Impact Factor & Total Articles & Total Cites\\
    \hline
      2014/2015 & 4.934         & 102            & 21017\\
      2013      & 5.466         & 154            & 20916\\
      2012      & 6.911         & 195            & 18840\\
      2011      & 6.810         & 118            & 16872\\
      2010      & 5.096         & 139            & 16971\\
      2009      & 4.878         & 129            & 17919\\
      2008      & 4.613         & 122            & 17993\\
    \tabucline[1pt]{-}
    \end{tabu}
  \end{center}
\end{frame}

\distributionofRT{ProceedingsOfTheIEEE}{Proceedings of the IEEE}{15 Days}{162.7923 Days}{564 Days}{182}

\distributionofPN{ProceedingsOfTheIEEE}{Proceedings of the IEEE}{7}{17.4809}{63}{182}

\relationshipbetweenRTPN{ProceedingsOfTheIEEE}{Proceedings of the IEEE}{182}

\begin{frame}{Four Kinds of Papers}
  There are 4 kinds of papers in Proceedings of the IEEE:
  \begin{itemize}
    \item \textbf{Prolog}: this kind of papers are not conventional papers, because prolog even has no reference.
    \item \textbf{Scanning Our Past}: this kind of papers are not conventional papers, either. there is one, and only one scanning our past in each issue.
    \item \textbf{Invited Paper}: this kind of papers are commonly written by the domain experts who are invited by the editor of Proceedings of the IEEE.
    \item \textbf{Contributed Paper}: as the name implies, this kind of papers have great contributions. If we submit paper to Proceedings of the IEEE, our paper must be a contributed paper.
  \end{itemize}
\end{frame}

\begin{frame}{Four Kinds of Papers}
  \begin{table}[htb]
    \centering
    \tabulinesep = 5pt
    \begin{tabu}to \textwidth{XX[-1, c]X[-1, c]}
      \tabucline[1pt]{-}
      Type              & Quantity & Proportion \\
      \hline
      Prolog            & 6        & $ 2.93\%$  \\
      Scanning our Past & 17       & $ 8.30\%$  \\
      Invited Paper     & 164      & $80.00\%$  \\
      Contributed Paper & 18       & $ 8.78\%$  \\
      \hline
      Total             & 205      & $100\%$    \\
      \tabucline[1pt]{-}
    \end{tabu}
  \end{table}
  We can see that $80.00\%$ papers are invited papers, so posting a contributed paper is quite difficult.
\end{frame}

\begin{frame}{Contributed Paper List}
  \begin{overlayarea}{\textwidth}{7cm}
  These 18 contributed papers are shown as following table.\vspace{10pt}

  \scriptsize
  \extrarowsep = 3mm
  \begin{tabu}{@{}X[-1, c, m]X[4, m]X[-1, m]@{}}
  \tabucline[1pt]{-}
  Number & Title & Comment\\
  \tabucline[1pt]{-}
  \only<+>{
  1  & Resource Allocation for Statistical Estimation                                                               & Simulation    \\\hline 
  2  & Reconfigurable Wireless Networks                                                                             & Case Study    \\\hline 
  3  & Engineering a Global Response to Infectious Diseases                                                         & Review Article\\\hline 
  4  & Radar Spectrum Engineering and Management: Technical and Regulatory Issues                                   & Experiment    \\\hline
  5  & Additively Manufactured Nanotechnology and Origami-Enabled Flexible Microwave Electronics                    & Implementation\\\hline
  6  & Spin-Based Computing: Device Concepts, Current Status, and a Case Study on a High-Performance Microprocessor & Case Study    \\\tabucline[1pt]{-}
  }
  \only<+>{
  7  & Software-Defined Networking: A Comprehensive Survey                                                          & Review Article\\\hline
  8  & Transparent Semiconducting Oxide Technology for Touch Free Interactive Flexible Displays                     & Simulation    \\\hline
  9  & Smart Agents in Industrial Cyber-Physical Systems                                                            & Review Article\\\hline
  10 & High-Speed Reconfigurable Circuits for Multirate Systems in SiGe HBT Technology                              & Experiment    \\\hline
  11 & Novel Polarization-Reconfigurable Converter Based on Multilayer Frequency-Selective Surfaces                 & Simulation    \\\hline
  12 & Aggregation and Charging Control of PHEVs in Smart Grid: A Cyber-Physical Perspective                        & Simulation    \\\tabucline[1pt]{-}
  }
  \only<+>{
  13 & The Cybersecurity Landscape in Industrial Control Systems                                                    & Review Article\\\hline
  14 & Key Issues With Printed Flexible Thin Film Transistors and Their Application in Disposable RF Sensors        & Implementation\\\hline
  15 & Reconfigurable Electromagnetics Through Metamaterials -- A Review                                            & Experiment    \\\hline
  16 & Microfluidic Stretchable Radio-Frequency Devices                                                             & Implementation\\\hline
  17 & High-power wind energy conversion systems: State-of-the-art and emerging technologies                        & Review Article\\\hline
  18 & Monolithic 3-D FPGAs                                                                                         & Implementation\\\tabucline[1pt]{-}
  }
  \end{tabu}
  \end{overlayarea}
\end{frame}

\subsection{IEEE Transaction on Industrial Informatics}
\begin{frame}{About Journal}
  \textbf{IEEE Transactions on Industrial Informatics} (ISSN: 1551-3203) focuses on knowledge-based factory automation as a means to enhance industrial fabrication and manufacturing processes. The other information is shown in the following table.\vspace{-10pt}
  \begin{center}
    \begin{tabu}to \textwidth{*4{X[-1, c]}}
    \tabucline[1pt]{-}
      Year      & Impact Factor & Total Articles & Total Cites\\
    \hline
      2014/2015 & --            &--              & --  \\
      2013      & 8.785         &231             & 2644\\
      2012      & 3.381         &92              & 969 \\
      2011      & 2.990         &71              & 739 \\
      2010      & 1.627         &63              & 328 \\
      2009      & 1.614         &39              & 287 \\
      2008      & 2.356         &28              & 227 \\
    \tabucline[1pt]{-}
    \end{tabu}
  \end{center}
\end{frame}

\distributionofRT{IEEETransactionOnIndustrialInformatics}{IEEE Transaction on Industrial Informatics}{1 Day}{245.3293 Days}{877 Days}{413}

\distributionofPN{IEEETransactionOnIndustrialInformatics}{IEEE Transaction on Industrial Informatics}{5}{11.4092}{18}{413}

\relationshipbetweenRTPN{IEEETransactionOnIndustrialInformatics}{IEEE Transaction on Industrial Informatics}{413}

\subsection{IEEE Transaction on Industrial Electronics}
\begin{frame}{About Journal}
  \textbf{IEEE Transactions on Industrial Electronics} (ISSN: 0278-0046) is published monthly. Its scope encompasses the applications of electronics, controls and communications, instrumentation and computational intelligence for the enhancement of industrial and manufacturing systems and processes. The other information is shown in the following table.\vspace{-10pt}
  \begin{center}
    \begin{tabu}to \textwidth{*4{X[-1, c]}}
    \tabucline[1pt]{-}
      Year      & Impact Factor & Total Articles & Total Cites\\
    \hline
      2014/2015 & 6.498         & 694            & 27141\\
      2013      & 6.500         & 553            & 24432\\
      2012      & 5.165         & 470            & 17404\\
      2011      & 5.160         & 531            & 15474\\
      2010      & 3.439         & 434            & 10294\\
      2009      & 4.678         & 505            & 10306\\
      2008      & 5.468         & 454            & 9014 \\
    \tabucline[1pt]{-}
    \end{tabu}
  \end{center}
\end{frame}

\distributionofRT{IEEETransactionOnIndustrialElectronics}{IEEE Transaction on Industrial Electronics}{11 Days}{204.7581 Days}{597 Days}{310}

\distributionofPN{IEEETransactionOnIndustrialElectronics}{IEEE Transaction on Industrial Electronics}{3}{10.5774}{16}{310}

\relationshipbetweenRTPN{IEEETransactionOnIndustrialElectronics}{IEEE Transaction on Industrial Electronics}{310}

\subsection{IEEE Transaction on Information Forensics and Security}
\begin{frame}{About Journal}
  \textbf{IEEE Transactions on Information Forensics and Security} (ISSN: 1556-6013) covers the sciences, technologies, and applications relating to information forensics, information security, biometrics, surveillance and systems applications that incorporate these features. The other information is shown in the following table.\vspace{-10pt}
  \begin{center}
    \begin{tabu}to \textwidth{*4{X[-1, c]}}
    \tabucline[1pt]{-}
      Year      & Impact Factor & Total Articles & Total Cites\\
    \hline
      2014/2015 & 2.408         & 144            & 2376\\
      2013      & 2.065         & 176            & 1598\\
      2012      & 1.895         & 157            & 1134\\
      2011      & 1.340         & 119            & 759 \\
      2010      & 1.725         & 81             & 682 \\
      2009      & 2.338         & 81             & 540 \\
      2008      & 2.230         & 68             & 265 \\
    \tabucline[1pt]{-}
    \end{tabu}
  \end{center}
\end{frame}

\distributionofRT{IEEETransactionOnInformationForensicsAndSecurity}{IEEE Transaction on Information Forensics and Security}{16 Days}{196.0488 Days}{428 Days}{123}

\distributionofPN{IEEETransactionOnInformationForensicsAndSecurity}{IEEE Transaction on Information Forensics and Security}{1}{12.9512}{27}{123}

\relationshipbetweenRTPN{IEEETransactionOnInformationForensicsAndSecurity}{IEEE Transaction on Information Forensics and Security}{123}

\subsection{Safety Science}
\begin{frame}{About Journal}
  \textbf{Safety Science} (ISSN: 0925-7535) serves as an international medium for research in the science and technology of human safety. It extends from safety of people at work to other spheres, such as transport, leisure and home, as well as every other field of man is hazardous activities. The other information is shown in the following table.\vspace{-10pt}
  \begin{center}
    \begin{tabu}to \textwidth{*4{X[-1, c]}}
    \tabucline[1pt]{-}
      Year      & Impact Factor & Total Articles & Total Cites\\
    \hline
      2014/2015 & 1.831         & 230            & 3959\\
      2013      & 1.672         & 221            & 3181\\
      2012      & 1.359         & 246            & 2393\\
      2011      & 1.402         & 159            & 1786\\
      2010      & 1.637         & 175            & 1788\\
      2009      & 1.220         & 153            & 1274\\
      2008      & 0.836         & 114            & 921 \\
    \tabucline[1pt]{-}
    \end{tabu}
  \end{center}
\end{frame}

\distributionofRT{SafetyScience}{Safety Science}{19 Days}{248.1364 Days}{972 Days}{308}

\distributionofPN{SafetyScience}{Safety Science}{4}{10.3766}{25}{308}

\relationshipbetweenRTPN{SafetyScience}{Safety Science}{308}

\subsection{Annual Reviews in Control}
\begin{frame}{About Journal}
  \textbf{Annual Reviews in Control} (ISSN: 1367-5788) covers the whole field of control and its applications. Most reviews are selected from the best reviews presented at meetings of IFAC, the International Federation of Automatic Control, re-written and broadened where necessary. The other information is shown in the following table.\vspace{-10pt}
  \begin{center}
    \begin{tabu}to \textwidth{*4{X[-1, c]}}
    \tabucline[1pt]{-}
      Year      & Impact Factor & Total Articles & Total Cites\\
    \hline
      2014/2015 & 2.518         & 20             & 957\\
      2013      & 1.878         & 28             & 788\\
      2012      & 1.289         & 28             & 662\\
      2011      & 1.319         & 21             & 482\\
      2010      & 1.884         & 24             & 410\\
      2009      & 1.886         & 23             & 441\\
      2008      & 1.109         & 20             & 365\\
    \tabucline[1pt]{-}
    \end{tabu}
  \end{center}
\end{frame}

\distributionofRT{AnnualReviewsInControl}{Annual Reviews in Control}{2 Days}{120.4706 Days}{727 Days}{102}

\distributionofPN{AnnualReviewsInControl}{Annual Reviews in Control}{7}{13.2255}{29}{102}

\relationshipbetweenRTPN{AnnualReviewsInControl}{Annual Reviews in Control}{102}